\chapter{Literature Review}
\label{ch:lit_rev}

\textbf{most research focuses on revenue management systems in the context of hotel or aviation. these differ from student accommodation because they focus on short term bookings}

\section{...}
Revenue management is defined as “the application of information systems and pricing strategies to allocate the right capacity to the right customer at the right price at the right time” \cite{Kimes2003HasAcceptable}. This is essential to a company selling a fixed number of rooms, setting the price too high would result in customers moving to competitors and cause reduced profits as well as a poor customer experience. Setting the price too low however may result in maximum occupancy but will trade this for potential losses in overall profits.

\vspace{5mm}

Booking cancellations is one of the main focuses in the revenue management industry as a whole \cite{Subramanian1999AirlineNo-shows}. Since a trade-off needs to be made between implementing a ridged cancellation policy, that imposes penalties on cancellation and a policy that has no impact to the customer. \cite{Jinhong2007ServiceCancellations} shows that implementing a ridged cancellation policy can act as a sales inhibitor and therefore reduce the number of customers. This however further increases the uncertainty as there is little impact to customers canceling bookings. \cite{Talluri2004TheManagement} finds that in some cases customers looking for the best price will make multiple bookings and then cancel all but one. 

\vspace{5mm}

\begin{itemize}
\item better explanation of comparison in student accommodation industry 
\end{itemize}

Looking at this problem specifically in the student accommodation industry, more focus needs to be directed towards capacity and price, this is unlike the hotel industry which deals more with short term stays. In the student accommodation the mean stay is [what is the mean stay] much longer and once the room is sold it is set at this fixed price for the duration of the academic year.
 \vspace{5mm}
 
 Forecasting demand is considered one of the five areas of revenue management problems the others are pricing, auctions, capacity control, and overbooking; \cite{Chiang2007AnResearch}  Development of a booking cancellation prediction model is in accordance to what was recognized by \cite{Chiang2007AnResearch} that revenue management should make use of mathematical and forecast models to take better advantage of the available data and technology.
 
\vspace{5mm}

As mentioned by \cite{RevenueWorldCat.org} and \cite{Weatherford2003AManagement} having the ability to accurately forecast demand is key to revenue management in determining how many rooms will be sold and to calculate the correct price at which to sell rooms. This ability to forecast demand then needs to be used in creating the correct pricing model "in which a perishable and nonrenewable set of resources satisfy stochastic price sensitive demand processes over a finite period of time" \cite{Bitran2003AnManagement}.

\vspace{5mm}

\begin{itemize}
\item state what PNR and PMS data is
\item reference on classification and on regression
\end{itemize}


Using machine learning to forecast bookings can be separated into two specific types of problems, regression problems where the aim is to predict a continuous quantitative value like the sum of bookings that will be cancelled in a given year and classification where each booking can be classified into two groups cancelled or not cancelled. Most of the research in this field is focused on solving it as a regression problem with  \cite{RomeroMorales2010ForecastingMining} stating “it is hard to imagine that one can predict whether a booking will be canceled or not with high accuracy simply by looking at PNR information”. PNR data however does not store the same number of data points that is possible with property management system data.

\vspace{5mm}

Taking advantage of what was recognized by \cite{Talluri2004TheManagement} that “science and technology now make it possible to manage demand on a scale and complexity that would be unthinkable through manual means”.  This study aims to use the tools now available in the field of machine learning and the increased number of data points on each customer to approach the prediction of cancellations as a classification problem.   

This backs up "Demand-based pricing is underused in many service industries" \cite{Kimes2003HasAcceptable} implying that a limited number of industries have properly taken advantage of new technology available in the field of machine learning. \textbf{As far as I can see this paper is currently the only one that looks at using machine learning for revenue management in the student accommodation industry}. 

\vspace{5mm}

\cite{Newell2018TheSector} stated "Amongst the alternative property sectors, student accommodation has recently become an important institutionalised property sector". In the UK alone the student accommodation industry is worth around 60 Billion \cite{UKWakefield} with an increase of 14 percent over the last 5 years \cite{UKWakefield}. Despite this there is no research in the field of forecasting or analysing demand. With the recent impact of COVID-19 causing a large number of students to have to study from home and in some cases no longer needing student accommodation it is more important than ever to have the ability to accurately forecast the demand. 



\vspace{5mm}

\begin{itemize}
\item Additive models

\item reference of saying it was not possible to predict using PMS data
\item comparing regression algorithms
\end{itemize}





\section{...}
...


\subsection{...}


\section{Summary}
...


