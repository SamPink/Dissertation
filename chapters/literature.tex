\chapter{Literature Review}
\label{ch:lit_rev}

\textbf{most research focuses on revenue management systems in the context of hotel or aviation. these differ from student accommodation because they focus on short term bookings}
\section{...}
Revenue management is defined as “the application of information systems and pricing strategies to allocate the right capacity to the right customer at the right price at the right time” (Kimes & Wirtz, 2003, p. 125). This is essential to a company selling a fixed number of rooms, setting the price too high would result in customers  moving to competitors and result is reduced profits as well as a poor customer experience. Setting the price too low however may result in maximum occupancy but will trade this for potential losses in overall profits.
Looking at this problem specifically in the student accommodation industry more focus needs to be directed towards capacity and price since unlike the hotel industry which deals more with short term stays. In the student accommodation the mean stay is [what is the mean stay] much longer and once the room is sold it is set at this fixed price for the duration of the academic year
The most simple method to forecast occupancy 


\section{...}
...


\subsection{...}


\section{Summary}
...


