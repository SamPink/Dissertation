\chapter{Introduction}

\section{Background}
Looking at renting in the student accommodation industry, specifically on the ability to accurately forecast revenue using the data stored in the property management system (\cite{Jain2006IntellectualPerspective}) by using machine learning to predict under what circumstances a contract will not be fulfilled.

\vspace{5mm}

A contract made in advanced between the student (customer) and the company defines the services that will be sold (the room) as well as the start and end date of the tenancy and the fix price the service will be sold at. In the case of student accommodation this contract is usually made a few months in advanced, in some cases before the student has finalized there plan to attend a specific university. Because of this the student withholds the right to cancel the contract before it is paid. The inherit uncertainty of the contract means there is no way for the company to definitively measure what the occupancy and sales of a given property will be in advanced since they have to take on the risk. This problem is not specific to the student accommodation industry, this is a problem of revenue management first identified by the aviation industry in 1966  (\cite{Chiang2007AnResearch})


\subsection{Problem Statement}

\begin{itemize}
\item What is the total proportion of bookings canceled and what is the overall lost revenue from this
\item total amount of money in bookings
\item total lost 
\end{itemize}


On a small scale it would be relatively easy for a human to identify the attributes common with bookings that would then be canceled, for example looking at the percentage of bookings canceled in the previous year. In this case however when the company operates over 70,000 beds across the world it is clear that no amount of human analysis would be able to take into account that many variables. 

\vspace{5mm}

\textbf{this paragraph is mostly a repeat of the 2nd paragraph}The fact that when a booking is made the student maintains the ability to cancel it up until a given time period for a given penalty means that it is the providers responsibility to account for the inherit uncertainty of the agreement. There's a number of different ways this problem can be solved that should take into account all of the possible variables, usually this is done by simply taking the number of bookings canceled in the previous year then using this number as an average to calculate the number of bookings likely to be canceled in the current year. The problem with this method is that it only takes into account one of the many variables (previous year figures) when in reality the business stores hundreds of data points on each of the bookings made.

It is for this reason I am attempting to solve this problem using machine learning


\begin{itemize}
\item  the business and all of the dataset will remain anonymous
\end{itemize}
    
\section{Aims and objectives}
The aim is to predict with some degree of certainty which of the bookings currently made and any new bookings will be cancelled by inputting all data points from all of the previous years bookings that have known outcomes, I hope to produce a model that can classify a booking made into 2 distinct states canceled or not canceled.

\begin{itemize}
\item Understanding why bookings are cancelled
\item Building a classification model
\end{itemize}


The first stage to any machine learning problem is obtaining accurate and clean data, in most cases (including this one) this is the hardest problem to solve as data is not usually stored in a way that makes it easy to read and analyse, most of the time data is stored  then never read used. For this reason I am approaching the problem by building a data warehouse to properly store and analyse the data before building a classification model [\textbf{reference supporting why accurate data is important}] 


\section{Solution approach}
\begin{itemize}
\item working with the company to understand what the requirements are, how cancellations predictions will be used, how results will be displayed to employees
\item why modeling needs to be fast and dynamic 
\item why building a data warehouse was important to solving the problem, since having the ability to retrain the new model is essential 
\item CRISP-DM, this is an iterating process of modeling an evaluation of the model
\item this entire process will be done in the azure cloud using technologies such a Python, Docker and Azure ML Studio 
\end{itemize}
